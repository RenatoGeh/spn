\documentclass[a4paper,10pt]{article}

\usepackage[brazilian]{babel}
\usepackage[utf8]{inputenc}
\usepackage[T1]{fontenc}
\usepackage{titlesec}
\usepackage{graphicx}
\usepackage{mathtools}
\usepackage{amsthm}
\usepackage[top=1.0in,bottom=1.0in]{geometry}
\usepackage{hyperref}
\usepackage[singlelinecheck=false]{caption}
\usepackage[backend=biber,url=true,doi=true,eprint=false]{biblatex}

\addbibresource{references.bib}

\newcommand\blfootnote[1]{%
  \begingroup
  \renewcommand\thefootnote{}\footnote{#1}%
  \addtocounter{footnote}{-1}%
  \endgroup
}

\titleformat{\section}
  {\normalfont\scshape\bfseries}{\thesection}{1em}{}
\titleformat{\subsection}
  {\normalfont\scshape\bfseries}{\thesubsection}{1em}{}
\titleformat{\paragraph}
  {\normalfont}{\theparagraph}{1em}{}
\titleformat{\subparagraph}
  {\normalfont}{\thesubparagraph}{1em}{}

\captionsetup[table]{labelsep=space}

\theoremstyle{plain}

\newtheorem*{spn-def}{Definição}

\title{\textbf{Aprendizado Automático de Sum-Product Networks (SPN)}}

\begin{document}
\date{}
\author{}
\vspace*{-40pt}
{\let\newpage\relax\maketitle}

Projeto de MAC0215 (Atividade Curricular em Pesquisa)

Aluno: Renato Lui Geh (Bacharelado em Ciência da Computação)

Orientador: Denis Deratani Mauá

\section{Introdução}

\paragraph{
  Para compreendermos melhor as aplicações de uma Sum-Product Network (SPN) precisamos primeiro entender distribuições
de probabilidade multivariadas, o que é o escopo de uma distribuição, as dificuldades de representar todas as 
probabilidades no espaço e o que é inferência de uma distribuição.
}

\subsection{Distribuições de probabilidade multivariadas}

\paragraph{
  Uma distribuição de probabilidade multivariada é uma distribuição de probabilidade que resulta na probabilidade 
de que cada variável aleatória $X_1,...,X_n$, para $n\geq{2}$, esteja em um dado intervalo ou conjunto para cada valor 
$x_i$, $1\leq{i}\leq{n}$ correspondente, sendo usadas em classificação de imagens em Sum-Product Networks. Se a 
distribuição possui apenas uma variável aleatória, dizemos que ela é monovariável. Segue abaixo um exemplo:
}

\paragraph{
  Consideremos um dado não viciado. Seja $A=1$ se o número tirado é par e $A=0$ caso contrário. Seja $B=1$ se o
mesmo número tirado em $A$ é primo e $B=0$ caso contrário. Podemos construir a tabela abaixo:
}

\begin{table}[h]
\caption{}
\begin{tabular}{l | *{6}{c}}
  & 1 & 2 & 3 & 4 & 5 & 6 \\
\hline
A & 0 & 1 & 0 & 1 & 0 & 1 \\
B & 0 & 1 & 1 & 0 & 1 & 0 \\
\end{tabular}
\end{table}

\paragraph{
  A distribuição de probabilidade multivariada resultante é:
}

\subparagraph{$P(A=0,B=0) = P\{1\} = \frac{1}{6}$ \\
  $P(A=1,B=0) = P\{4,6\} = \frac{2}{6}$ \\
  $P(A=0,B=1) = P\{3,5\} = \frac{2}{6}$ \\
  $P(A=1,B=1) = P\{2\} = \frac{1}{6}$
} 

\subsection{Escopo de uma distribuição}

\paragraph{
  Denotamos $P(X_1=x_1,...,X_n=x_n)$, dados valores $x_1,...,x_n$, uma distribuição de probabilidade multivariada
com variáveis aleatórias $X_1,...,X_n$. Dada uma distribuição de probabilidade, chamamos de escopo da distribuição 
o conjunto de variáveis que definem o domínio da função, ou seja, $X_1,...,X_n$. No caso da Tabela 1, o escopo da 
probabilidade é o conjunto $\{A, B\}$. 
}

\begin{spn-def} Sejam $P_1 (X_1,...,X_k)$ e $P_2 (X_p,...,X_n)$ distribuições de probabilidade multivariadas cujos
  escopos têm mesmo tamanho.
\begin{enumerate} \itemsep0pt
  \item Dizemos que $P_1$ e $P_2$ tem mesmo escopo se para toda variável aleatória $X_i$ em $P_1$ existe uma
variável aleatória $X_j$ em $P_2$ tal que $i=j$.
  \item Dizemos que $P_1$ e $P_2$ tem escopos disjuntos se para qualquer variável aleatória $X_i$ em $P_1$,
toda variável aleatória $X_j$ em $P_2$ obedece a regra $i\neq{j}$. 
\end{enumerate}
\end{spn-def}

\subsection{Espaço de uma distribuição multivariada}

\paragraph{
  Vamos reutilizar a Tabela 1, renomeando $A$ para $X_1$ e $B$ para $X_2$. Portanto teremos $P(X_1=x_1,X_2=x_2)$. 
Montando a tabela temos que:
}

\begin{table}[h*]
\caption{}
\begin{tabular}{*{3}{c}}
$X_1$ & $X_2$ & $P(X_1=x_1,X_2=x_2)$ \\
\hline
$X_1=0$ & $X_2=0$ & $\frac{1}{6}$ \\
$X_1=0$ & $X_2=1$ & $\frac{2}{6}$ \\
$X_1=1$ & $X_2=0$ & $\frac{2}{6}$ \\
$X_1=1$ & $X_2=1$ & $\frac{1}{6}$ \\
\end{tabular}
\end{table}

\paragraph{
  Pode-se ver que uma tabela com todas as possíveis probabilidades de todas as variáveis terá $2^n$ elementos, 
onde $n$ é o número de variáveis aleatórias. Portanto o crescimento em espaço necessário é exponencial, o que
é impraticável em conjuntos de dados maiores. A solução vem com o uso de Redes Profundas (ex.: Sum-Product Networks), 
que tornam $O(2^n)$ em $O(n)$ no tamanho do grafo.
}

\subsection{Inferência}

\paragraph{
  Diz-se inferência a probabilidade de uma distribuição multivariada dados valores fixos de certas
variáveis aleatórias. Formalmente, dado um conjunto $e$ de evidência que representa os valores 
$X_{i_1}=x_{i_1},...,X_{i_k}=x_{i_k}$, a inferência de $P(X_1,...,X_n)$ dado $e$ é a probabilidade
normalizada $P(x)=\frac{P'(x)}{Z}$, onde $Z$ é a função partição que normaliza $P'(x)$ em uma 
distribuição de probabilidade e $P'(x)$ é a função não normalizada de $P(X_1,...,X_n)$ dado
evidência $e$ que representa a inferência. Em Sum-Product Networks, $Z=S(*)=S(1,...,1)$. 
}

\paragraph{
  Como um exemplo, consideremos o caso em que as variáveis aleatórias representam variáveis 
Booleanas. Seja o conjunto evidência $e=\{X_1=1,X_2=0\}$ numa distribuição $P(X_1,X_2,X_3,X_4)=
P(x_1,x_2,x_3,x_4,\overline{x_1},\overline{x_2},\overline{x_3},\overline{x_4})$, associamos todas 
as ocorrências $x_i \in e$ às suas equivalentes em $P$ e o resto a $0$. Para as variáveis que não 
fixamos, associamos a $1$. Então, a inferência é $P(x_1=1,x_2=0,x_3=1,x_4=1,\overline{x_1}=0,
\overline{x_2}=1,\overline{x_3}=1,\overline{x_4}=1)$, que representa todas as combinações de 
probabilidade de $X_1,...,X_4$ dados eventos conhecidos $X_1=1$ e $X_2=0$.
}

\section{Definição de uma Sum-Product Network}

\paragraph{
  Sum-Product Networks (SPN) são uma nova classe de modelos probabilísticos cuja inferência é sempre
polinomial no tamanho da rede.
}

\begin{spn-def} Pela definição recursiva de Gens e Domingos\cite{gens-domingos}:
\begin{enumerate} \itemsep0pt
  \item Uma distribuição monovariável polinomial é uma SPN.
  \item Um produto de SPNs cujos escopos são disjuntos é uma SPN.
  \item Uma soma de SPNs com peso não negativo cujos elementos tem mesmo escopo é uma SPN.
  \item Nada mais é uma SPN.
\end{enumerate}
\end{spn-def}

\paragraph{
  Podemos definir graficamente uma SPN como um grafo direcionado acíclico (DAG) onde as 
folhas são sempre variáveis (ou distribuições monovariáveis), seus nós internos são somas ou 
produtos e para todo vértice conectando um nó soma com um nó filho há um peso não negativo. 
Também assume-se que toda soma e produto estão em alturas alternantes, ou seja, todo nó pai 
de um nó interno que é soma é um produto e vice-versa.
}

\begin{figure}[h]
\centering\includegraphics[scale=0.7]{imgs/domingos_poon.jpg}
\caption{Um exemplo de uma SPN com variáveis Booleanas, onde $x_1,...,x_d$ e 
  $\overline{x_1},...,\overline{x_d}$ são folhas e o resto dos nós são somas ou produtos.\cite{poon-domingos}}
\end{figure}

\paragraph{
  Tomando a Figura 1 como exemplo, considere variáveis aleatórias binárias $X_1,...,X_n \in \{0,1\}$. Para este
exemplo $n=3$. Para todo $X_i$, indicamos por $x_i$ o indicador em que $X_i=1$ e $\overline{x}_i$ para quando
$X_i=0$. Portanto, quando $X_i=1$, $x_i=1$ e $\overline{x}_i=0$; e quando $X_i=0$, $x_i=0$ e $\overline{x}_i=1$.
A função multilinear do diagrama que representa a SPN é a seguinte:
} 

\begin{equation}
\begin{split}
S(X_1,X_2,X_3) & =\\
& =S(x_1,x_2,x_3,\overline{x}_1,\overline{x}_2,\overline{x}_3)=\\
& =0.95x_1(0.1x_2+0.9\overline{x}_2)(0.5x_3+0.5\overline{x}_3)+\\
&\phantom{=} +0.05\overline{x}_1(0.2x_2+0.8\overline{x}_2)(0.3x_3+0.7\overline{x}_3)
\end{split}
\end{equation}

\paragraph{
  Quando todos indicadores pertencentes ao escopo da SPN $S$ são iguais a $1$, diz-se que $S$ é
$S(*)$ e que ela é a função partição. $S(x)/S(*)$ é a distribuição de probabilidade normalizada da SPN $S$.
}

\section{Características}

\paragraph{
  Há várias vantagens de SPNs sobre outras redes de aprendizado:
}

\begin{enumerate} \itemsep0pt
  \item SPNs têm estrutura parecida a outros Modelos Gráficos Probabilisticos (PGM), mas a 
    representação de certos tipos de independência é mais fácil.
  \item SPNs têm inferência polinomial no tamanho do grafo, enquanto que inferência em Redes Bayesianas
    é NP-difícil.
  \item Experimentos mostram que aprendizado de arquiteturas SPN tiveram melhores resultados quando
    comparadas a arquiteturas estáticas.\cite{clustering}
\end{enumerate}

\newpage

\section{Aplicações}

SPNs obtiveram resultados impressionantes em muitos conjuntos de dados\cite{website:spn-uwashington}, tais como:

\begin{itemize} \itemsep0pt
  \item Reconstrução de imagens.
  \item Classificação.
  \item Reconhecimento de atividade.
  \item Logs click-through.
  \item Sequências de ácido nucleico.
  \item Filtragem colaborativa.
\end{itemize}

\section{Proposta}

\paragraph{
  O objetivo deste projeto é realizar uma comparação das principais técnicas de aprendizado 
automático de Sum-Product Network a partir de um conjunto de dados, avaliando o desempenho 
e os resultados das técnicas utilizadas.
}

\paragraph{
  Será estudada a estrutura e propriedades de uma Sum-Product Network e em seguida serão
aplicadas as seguintes técnicas de aprendizado:
}

\begin{itemize} \itemsep0pt
  \item Divisão em subconjuntos mutualmente independentes e EM clustering.\cite{gens-domingos}  
  \item Busca gulosa.\cite{greedy-search}
  \item Aprendizado com clustering de variáveis.\cite{clustering}
  \item Aprendizado por meio de SPNs Bayesianas Não-Paramétricas.\cite{non-parametric-bayesian}
\end{itemize}

\paragraph{
  Depois o aluno irá comparar as técnicas implementadas e apresentar os resultados.
}

\section{Acompanhamento}

\paragraph{
  Relatórios semanais serão publicados no seguinte endereço: 
}

\subparagraph{\url{http://www.ime.usp.br/~renatolg/mac0215/doc/reports/}}

\paragraph{
  É também possível acompanhar tanto os relatórios quanto as implementações pelo repositório
do projeto:
}

\subparagraph{\url{https://github.com/RenatoGeh/mac0215/}}

\newpage

\section{Referências}

\printbibliography[title={Artigos},type=article]
\printbibliography[title={Websites},type=misc]

\blfootnote{Os artigos listados na referência acima podem ser encontrados em: 
  \url{http://www.ime.usp.br/~renatolg/mac0215/articles/}}

\end{document}
