\documentclass[a4paper,10pt]{article}

\usepackage[english]{babel}
\usepackage[utf8]{inputenc}
\usepackage{titlesec}
\usepackage{graphicx}
\usepackage{mathtools}
\usepackage{amsthm}
\usepackage{amsfonts}
\usepackage[top=1.0in,bottom=1.0in]{geometry}
\usepackage{hyperref}
\usepackage[singlelinecheck=false]{caption}
\usepackage[backend=biber,url=true,doi=true,eprint=false]{biblatex}
\usepackage{enumitem}
\usepackage[x11names, rgb]{xcolor}
\usepackage{tikz}
\usepackage[justification=centering]{caption}
\usepackage{indentfirst}
\usepackage{abstract}
\usepackage[titletoc]{appendix}

\usetikzlibrary{snakes,arrows,shapes}

\addbibresource{../common/references.bib}

\newcommand\blfootnote[1]{%
  \begingroup
  \renewcommand\thefootnote{}\footnote{#1}%
  \addtocounter{footnote}{-1}%
  \endgroup
}

\DeclareMathOperator*{\argmin}{arg\,min}
\DeclareMathOperator*{\argmax}{arg\,max}

\newcommand\defeq{\mathrel{\overset{\makebox[0pt]{\mbox{\normalfont\tiny\sffamily def}}}{=}}}

\renewcommand{\abstractnamefont}{\normalfont\Large\bfseries}

\titleformat{\section}
  {\normalfont\scshape\bfseries}{\thesection}{1em}{}
\titleformat{\subsection}
  {\normalfont\scshape\bfseries}{\thesubsection}{1em}{}
\titleformat{\paragraph}
  {\normalfont}{\theparagraph}{1em}{}
\titleformat{\subparagraph}
  {\normalfont}{\thesubparagraph}{1em}{}

\captionsetup[table]{labelsep=space}

\theoremstyle{plain}

\newtheorem*{spn-def}{Definition}
\newtheorem*{spn-thm}{Theorem}

\setlength{\parskip}{1em}

\begin{document}

\begin{titlepage}
  \begin{center}
    \LARGE
    \textbf{An Introduction to Sum-Product Networks}

    \vspace{1.7cm}
    \Large
    A collection of studies on properties, structure, inference and learning on Sum-Product
    Networks

    \vspace{1.7cm}
    \large
    Student: Renato Lui Geh

    Supervisor: Denis Deratani Mauá (DCC IME-USP)
    \vfill
    \large
    University of São Paulo / Universidade de São Paulo (USP)

    Institute of Mathematics and Statistics / Instituto de Matemática e Estatística (IME)
    \vspace{1.5cm}
  \end{center}
\end{titlepage}

\newpage
\null\vspace{\fill}
\begin{abstract}
  \large
  This work is a collection of ongoing studies I am working on for my undergraduate research
  project on automatic learning of Sum-Product Networks. The main objective of this work is logging
  my study notes on this subject in an instructive and uncomplicated way. Most scientific papers
  are cluttered with intricate names and require extensive background on the subject in order for
  the reader to understand what is going on. In this paper we seek to provide an easy reference and
  introductory reading material to those who intend to work with Sum-Product Networks.

  This study is divided into five main sections. We start with an introductory section regarding
  probabilistic graphical models and why Sum-Product Networks are so interesting. Next we talk
  about the structure of the model. Thirdly, we analyse some properties and theorems. Fourthly, we
  look on how to perform exact tractable inference. And finally we take a look at how to perform
  learning.
\end{abstract}
\vspace{\fill}
\newpage
\large
\tableofcontents
\normalsize
\newpage

\section{Introduction}

We assume the notation~\ref{app:not} and mathematical background~\ref{app:bak} defined in the
Appendix.

\newpage
\begin{appendices}
  \section{Notation}\label{app:not}
  Notation bwa
  \section{Mathematical background}\label{app:bak}
\end{appendices}

\printbibliography{}

\end{document}
