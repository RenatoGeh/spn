\documentclass[a4paper,10pt]{article}

\usepackage[brazilian]{babel}
\usepackage[utf8]{inputenc}
\usepackage{titlesec}
\usepackage{graphicx}
\usepackage{mathtools}
\usepackage{amsthm}
\usepackage{amsfonts}
\usepackage[top=1.0in,bottom=1.0in]{geometry}
\usepackage{hyperref}
\usepackage[singlelinecheck=false]{caption}
\usepackage[backend=biber,url=true,doi=true,eprint=false]{biblatex}
\usepackage{enumitem}
\usepackage[x11names, rgb]{xcolor}
\usepackage{tikz}
\usepackage[justification=centering]{caption}
\usepackage{indentfirst}
\usetikzlibrary{snakes,arrows,shapes}

\addbibresource{../common/references.bib}

\newcommand\blfootnote[1]{%
  \begingroup
  \renewcommand\thefootnote{}\footnote{#1}%
  \addtocounter{footnote}{-1}%
  \endgroup
}

\DeclareMathOperator*{\argmin}{arg\,min}
\DeclareMathOperator*{\argmax}{arg\,max}

\newcommand\defeq{\mathrel{\overset{\makebox[0pt]{\mbox{\normalfont\tiny\sffamily def}}}{=}}}

\titleformat{\section}
  {\normalfont\scshape\bfseries}{\thesection}{1em}{}
\titleformat{\subsection}
  {\normalfont\scshape\bfseries}{\thesubsection}{1em}{}
\titleformat{\paragraph}
  {\normalfont}{\theparagraph}{1em}{}
\titleformat{\subparagraph}
  {\normalfont}{\thesubparagraph}{1em}{}

\captionsetup[table]{labelsep=space}

\theoremstyle{plain}

\newtheorem*{spn-def}{Definition}
\newtheorem*{spn-thm}{Theorem}

\setlength{\parskip}{1em}

\title{\textbf{Learning Probabilistic Models}}

\begin{document}
\date{}
\author{
Renato Lui Geh \\
Supervisor: Denis Deratani Mauá
}
\vspace*{-40pt}
{\let\newpage\relax\maketitle}

\section{Introduction}

In this report we cite a few probabilistic models and how to perform learning on them. This report
will be divided into five sections:

\begin{enumerate}
  \item Inference on Bayesian networks
  \item Bayesian learning
  \item Maximum-likelihood parameter learning
  \item Naive Bayes
  \item Bayesian parameter learning
\end{enumerate}

On (1) we will cover the definition of inference on Bayesian networks and how to perform simple
queries. We will cover exact inference (which is in most cases intractable) and cite sources for
approximate inference.

Next, on (2) we will take a look at the most simple of learning -- Bayesian learning. We will show
how to perform a simple query using the candy example presented on \textit{Artificial Intelligence:
A Modern Approach}\cite{aima} by Russel and Norvig. We will use this example throughout most of
these learning methods.

We will then present Maximum-likelihood parameter learning (3), a better form of learning that is
widely used in many applications. However this method performs poorly on small training sets.

Fourthly we will present Naive Bayes, a well known model that uses a simple Bayesian network
composed of a single parent node and $n$ children nodes that are d-separated (i.e. independent) of
each other given their parent.

Finally, we will show how to perform Bayesian parameter learning. In this method we learn the
parameters of a compiled arithmetic circuit that represents a Bayesian network.

This report is heavily influenced by the works of Russel and Norvig on AIMA\cite{aima}. It is based
on the readings of chapter 14 (Probabilistic Reasoning) and 20 (Learning Probabilistic Models) of
the 3rd edition.

We consider the reader has some knowledge of basic probability and a basic understanding of
Bayesian networks. More specifically we consider the reader has read the following previous
reports: \cite{project-def, report-1, report-2, report-5, report-8, report-9}.

\subsection{Notation}

For this and all following reports we will use the notations described in this subsection.

Any function that is of the form $P(\cdot)$ is a probability function. $\mathbf{X}$ denotes a set
of variables with instances $X_1,...,X_k$. $X$ denotes a random variable. $\mathbf{E}$ is the set
of evidence variables and $\mathbf{e}$ is an observed event. A prior probability is a probability
function of the form $P(\mathbf{X})$. A posterior probability is of the form $P(\mathbf{X}|\mathbf{
Y})$. A probability function $\mathbf{P}(\cdot)$ returns a distribution of probability.

\section{Inference on Bayesian networks}

Inference in any probabilistic model is to compute a posterior probability distribution of a query
of variables given an observed evidence. Let $\mathbf{Y}$ be a set of nonevidence nonquery
variables $Y_1,...,Y_l$ that we call hidden variables, the complete set of variables is $\mathbf{X}
= \{X\} \cup \mathbf{E} \cup \mathbf{Y}$, where $X$ is the query variable. Then the inference query
of $X$ is $\mathbf{P}(X|\mathbf{e})$.

Any conditional probability can be computed by summing the factors from the full joint
distribution. That is, $\mathbf{P}(X|\mathbf{e})$ can be computed by summing all the possible
events of the hidden variables of the joint distribution.

\begin{equation}
  \mathbf{P}(X|\mathbf{e}) = \alpha \mathbf{P}(X, \mathbf{E}) = \alpha \sum_\mathbf{y} \mathbf{P}(
  X, \mathbf{e}, \mathbf{y})
\end{equation}

Thus, we can use a Bayesian network to represent the full joint distribution. Furthermore, since:

\begin{equation}
  P(x_1,...,x_n) = \prod_{i=1}^n P(x_i|Parents(X_i))
\end{equation}

Then we can write $P(x,\mathbf{e},\mathbf{y})$ as products of the conditional probabilities of the
Bayesian network. Because of that, we can compute a query via the sums of products of the
conditional probabilities of the network. Report \cite{report-1} shows how to represent a Bayesian
network with sums and products.

After we have the resulting expression given by equations 1 and 2, we can use factor properties
\cite{report-2} to simplify the expression and avoid repeated calculations \cite{report-5}.

Since exact inference with Bayesian networks is, in the general case, intractable, it makes these
methods in the real world unrealistic. Section 14.5 from \cite{aima} describes some methods for
approximate inference. Darwiche (2009)\cite{bayes-net-darwiche} is also a great source for Bayes
nets.

\section{Bayesian learning}



\newpage

\printbibliography

\end{document}
