\documentclass[a4paper,10pt]{article}

\usepackage[brazilian]{babel}
\usepackage[utf8]{inputenc}
\usepackage{titlesec}
\usepackage{graphicx}
\usepackage{mathtools}
\usepackage{amsthm}
\usepackage{amsfonts}
\usepackage[top=1.0in,bottom=1.0in]{geometry}
\usepackage{hyperref}
\usepackage[singlelinecheck=false]{caption}
\usepackage[backend=biber,url=true,doi=true,eprint=false]{biblatex}
\usepackage{enumitem}
\usepackage[x11names, rgb]{xcolor}
\usepackage{tikz}
\usepackage[justification=centering]{caption}
\usepackage{indentfirst}
\usetikzlibrary{snakes,arrows,shapes}

\addbibresource{../reports/common/references.bib}

\newcommand\blfootnote[1]{%
  \begingroup
  \renewcommand\thefootnote{}\footnote{#1}%
  \addtocounter{footnote}{-1}%
  \endgroup
}

\DeclareMathOperator*{\argmin}{arg\,min}
\DeclareMathOperator*{\argmax}{arg\,max}

\newcommand\defeq{\mathrel{\overset{\makebox[0pt]{\mbox{\normalfont\tiny\sffamily def}}}{=}}}

\titleformat{\section}
  {\normalfont\scshape\bfseries}{\thesection}{1em}{}
\titleformat{\subsection}
  {\normalfont\scshape\bfseries}{\thesubsection}{1em}{}
\titleformat{\paragraph}
  {\normalfont}{\theparagraph}{1em}{}
\titleformat{\subparagraph}
  {\normalfont}{\thesubparagraph}{1em}{}

\captionsetup[table]{labelsep=space}

\theoremstyle{plain}

\newtheorem*{spn-def}{Definição}
\newtheorem*{spn-thm}{Teorema}

\setlength{\parskip}{1em}

\title{\textbf{Aprendizado Automático de Sum-Product Networks (SPNs)}}

\begin{document}
\date{}
\author{}
\vspace*{-40pt}
{\let\newpage\relax\maketitle}

Aluno: Renato Lui Geh (Bacharelado em Ciência da Computação)

Supervisor: Denis Deratani Mauá

\section{Introdução}

O objetivo deste projeto de Iniciação Científica é utilizar Aprendizado de Máquina para aprender
automaticamente a estrutura de um modelo probabilístico denominado Sum-Product Network (SPN).

Modelos probabilísticos baseados em grafos têm como objetivo representar distribuições de
probabilidade de forma compacta.

Para extraír conhecimento de um modelo probabilístico, computa-se inferência. Inferência na
maioria dos modelos gráficos é intratável, já que o número de termos na distribuição é exponencial.

Existem modelos gráficos que possuem inferência tratável, porém a maioria não consegue representar
de forma compacta e geral uma distribuição. A maioria dos PGMs solucionam o problema da
intractabilidade computando a inferência aproximada.

Sum-Product Networks são PGMs que, quando completas e consistentes, computam a inferência exata e
em tempo tratável. Adicionalmente, SPNs se mostraram mais gerais que outros modelos que computam
inferência em tempo tratável.\cite{poon-domingos}

Como aprendizado de uma SPN depende da inferência, a intractabilidade do aprendizado depende da
intractabilidade da inferência.

\section{Objetivos}

Neste projeto de Iniciação Científica, o aluno irá estudar os seguintes tópicos:

\begin{itemize}
  \item Propriedades e estrutura de uma Sum-Product Network.
  \item Inferência em SPNs.
  \item Aprendizado:
  \begin{itemize}
    \item Dos pesos de uma SPN.\cite{poon-domingos}
    \item Da estrutura de uma SPN.\cite{gens-domingos}
    \item Por busca gulosa.\cite{greedy-search}
    \item Por clustering de variáveis.\cite{clustering}
    \item Por SPNs bayesianas não-paramétricas.\cite{non-parametric-bayesian}
  \end{itemize}
\end{itemize}

\section{Cronograma}

O aluno deverá reservar 10 horas por semana para estudos relacionados à IC. Além disso, o aluno
irá escrever relatórios semanais do que foi estudado na semana. Os relatórios estarão disponíveis
em:

\subparagraph{\url{http://www.ime.usp.br/~renatolg/spn/doc/reports/}}

Tanto os relatórios quanto as implementações estarão disponíveis pelo repositório do projeto:

\subparagraph{\url{https://github.com/RenatoGeh/spn/}}

\newpage

\printbibliography

\end{document}
